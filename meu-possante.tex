\documentclass[12pt]{article}

\usepackage{sbc-template}

\usepackage{graphicx,url}

\usepackage[brazil]{babel}
\usepackage[utf8]{inputenc}


\sloppy

\title{Meu Possante: Um Aplicativo de Auxílio a Mecânica Automotiva}

\author{José Lucas dos S. Borges{\inst1}}

\address{Departamento de Ciência da Computação -- Instituto de Matemática\\
    Universidade Federal da Bahia (UFBA)\\
    Rua Barão de Jeremoabo, s/n - Ondina, Salvador - BA, 40170-115
\email{jlucas@dcc.ufba.br}
}

\begin{document}

\maketitle

\begin{abstract}
    This paper proposes a Decision Support System (DSS) in the format of an
    mobile application in the automotive mechanics area, and it is aimed for
    lay people in this subject. Here are mentioned and discussed the variables
    used in the application, the types of Information System (IS) and DSS in
    which it fits, the main algorithms used and a brief system modelling, with
    some screen prototypes.
\end{abstract}

\begin{resumo}
    Este artigo propõe um Sistema de Apoio a Decisão (SAD) no formato de um
    aplicativo para celulares na área de mecânica automotiva para pessoas leigas
    no assunto. Aqui são apresentados e discutidos as variáveis utilizadas no
    aplicativo, os tipos de Sistema de Informação (SI) e SAD nos quais ele se
    enquadra, os principais algoritmos utilizados e uma breve modelagem do
    sistema, com alguns protótipos de tela.
\end{resumo}


\section{Introdução} \label{sec:introducao}

Mecânica automotiva é um assunto extremamente complexo. Cada um dos inúmeros
componentes do automóvel tem seu próprio ciclo de vida e devem ser revisados
e trocados em seu próprio tempo. Além disso, certas condições de uso podem
diminuir a vida útil de algumas peças, tornando mais frequente a necessidade
de revisão.

A falta de conhecimento destes fatos pode levar ao dono de um automóvel
negligenciar as manutenções no tempo correto, causando desde transtornos
que poderiam ser facilmente evitados a acidentes ocasionados por falhas
mecânicas.

O uso de um aplicativo para celular pode ser uma grande ajuda na decisão de
quando é necessário a revisão e troca de peças, alertando o usuário
visualmente quando alguma manutenção está próxima.

\section{Referencial Teórico} \label{sec:referencialteorico}
Esta seção aborda os principais fundamentos teóricos envolvidos na notificação
oportuna de motoristas, tema central deste trabalho. Aqui são discutidos X Y e Z.

Para descobrir quais são os principais fatores que influenciam na frequência de
manutenção de automóveis, foi feita uma busca nos manuais dos carros mais vendidos
no Brasil em 2016 que, segundo a Federação Nacional da Distribuição de Veículos
Automotores - Fenabrave - foram o Hyundai HB20, Chevrolet Onix e Ford Ka Hatch
\cite{fenabrave}.

Todos os manuais citam três fatores importantes que influenciam na frequência
de manutenção de peças veiculares: a quilometragem do veículo, suas condições de
uso e o tempo desde a última revisão. \cite{manualhyundai, manualonix, manualka}


\section{Algoritmos Utilizados} \label{sec:algoritmos}


\section{Modelagem do Sistema} \label{sec:modelagem}

\bibliography{meu-possante}
\bibliographystyle{sbc}

\end{document}
